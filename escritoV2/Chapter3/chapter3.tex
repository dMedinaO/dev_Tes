%!TEX root = ../thesis.tex
%*******************************************************************************
%****************************** Third Chapter **********************************
%*******************************************************************************
\chapter{Digitalización de secuencias lineales de proteínas aplicadas al reconocimiento de patrones y modelos predictivos \label{cap3}}

% **************************** Define Graphics Path **************************
\ifpdf
    \graphicspath{{Chapter3/Figs/Raster/}{Chapter3/Figs/PDF/}{Chapter3/Figs/}}
\else
    \graphicspath{{Chapter3/Figs/Vector/}{Chapter3/Figs/}}
\fi

Desarrollar modelos predictivos basados en algoritmos de aprendizaje supervisado, o, la identificación de patrones aplicando técnicas de clustering, son tareas muy relevantes a la hora de trabajar con secuencias de proteínas, ya sea para identificar grupos con características comunes o entrenar modelos predictivos de respuestas de interés. En ambos casos, se requiere el uso de conjuntos de datos altamente informativos y con características numéricas para poder utilizar los métodos implementados en las librerías actuales \cite{pedregosa2011scikit}.

Diferentes metodologías se han implementado, para manipular las variables categóricas en set de datos y lograr su codificación numérica. Enfoques basados en adición de columnas según las categorías o simple transformación empleando representaciones en conjuntos naturales, suelen ser utilizados. No obstante, generan bastante discusión sobre las nuevas representaciones. y a su vez, el hecho de aumentar el número de columnas, conlleva a incrementar las dimensiones del conjunto de datos, provocando efectos en los desempeños de los algoritmos \cite{pedregosa2011scikit}. 

Particularmente, en secuencias de proteínas, se han utilizado las frecuencias de incidencia de los residuos para codificarlos, la cual, pese a su simplicidad, ha resultado ser efectiva en diferentes casos de uso \cite{ozbudak2014protein}. No obstante, este tipo de codificación, no permite explorar el ambiente bajo el cual se encuentran los residuos y tampoco considera el efecto de propiedades fisicoquímicas ni termodinámicas.

En diferentes estudios, los residuos se describen a partir de sus propiedades fisicoquímicas y adicional a ello, se emplea información que permite describir el ambiente del residuo a caracterizar, empleando binarizaciones para describir los residuos cercanos, ya sea por medio del uso de un rango espacial, utilizando modelos o estructuras tridimensionales en donde se representan las coordenadas espaciales de los residuos, o empleando un rango lineal en secuencias lineales de proteínas \cite{capriotti2005mutant2, capriotti2008three}.

Un enfoque basado en las propiedades fisicoquímicas en combinación con la aplicación de transformaciones de Fourier, ha permitido demostrar que ciertos residuos permiten entregar las características asociadas a la propiedad en estudio, además, facilita comprender el aporte del ambiente sobre estos y representa una forma de estudio novedosa para el uso de información de secuencias lineales. Siendo una metodología ampliamente utilizada para identificar residuos que aporten a la propiedad, por medio de la representación de señales asociadas al espacio de frecuencias \cite{veljkovic1985possible, cosic2016analysis, cadet2018application}.

A pesar de ser una metodología interesante a la hora de estudiar secuencias lineales, exhiben problemas notorios sobre la selección de las propiedades relevantes a analizar, ya que, existe un número considerablemente alto de propiedades posibles a utilizar, descritas principalmente en las base de datos AAIndex \cite{Kawashima2000}, y es factible que diferentes familias de proteínas exhiban comportamientos notoriamente no similares y diverjan en cuanto a las propiedades que puedan ser representativas, inclusive, a la hora de estudiar mutaciones en una misma proteína puede que no sólo una propiedad permita su caracterización, si no, que un conjunto pequeño de éstas \cite{cadet2018application}.

En el presente capítulo, se exponen en detalle, diferentes formas de representar secuencias lineales de proteínas, seguido a su vez del planteamiento del uso de transformadas de Fourier para la digitalización de propiedades fisicoquímicas y cómo es posible utilizar éstas para la identificación de patrones en secuencias lineales o el desarrollo de modelos de clasificación/regresión y la exposición de casos de uso en diferentes proteínas de interés. 

\section{Metodologías asociadas a la codificación de variables categóricas}

Diferentes metodologías existen para poder codificar variables categóricas, a su vez, para set de datos de proteínas con secuencias lineales, es factible utilizar sus propiedades fisicoquímicas o frecuencias de residuos. Las principales metodologías usadas a la fecha se expone a continuación.

\subsection{One Hot encoder}

One Hot encoder, es una de las técnicas más utilizadas a la hora de codificar variables categóricas y se basa principalmente en la adición de columnas con respecto a las categorías existentes en un conjunto de datos \cite{brownlee2017one}.

Dado el vector  $x$ de tamaño $n$ con $m$ categorías, por definición, One Hot encoder agrega al conjunto de datos $m$ columnas, tal que, por cada categoría se adiciona una nueva columna al set de datos. Las nuevas columnas se completan con una binarización de los elementos, indicando si el elemento $x_{i}$ posee la categoría $m_{j}$ con un valor 1 y en caso contrario 0. Es posible expresar esto como se expone a continuación.

Sea $x$ vector de $m$ categorías de dimensiones
$n \times 1$ representado por
\[ x = \left( \begin{array}{ccc}
x_{0}\\
x_{1}\\
\vdots\\
x_{n-1} \end{array} \right)\] 
Su codificación mediante One Hot Encoder corresponde al vector x'(x)
\[ x'(x) = \left( \begin{array}{ccccc}
1 & 0 & 0 & \cdots & 0 \\
0 & 1 & 0 & \cdots & 0 \\
\vdots\\
0 & 0 & 0 & \cdots & 1 \end{array} \right) \] 

Esta metodología, si bien es altamente usada, implica que, a medida que aumentan la cantidad de categorías incrementan el número de columnas a agregar en el set de datos, es decir, si se tienen $m$ categorías se adicionan $m$ columnas. Esto último, puede provocar que los set de datos se afecten por problemas relacionados con la \textit{maldición de dimensionalidad} \footnote{Ver sección \ref{problemas}}, ya que, a medida que aumentan los descriptores, aumenta la probabilidad de que estos no sean informativos, provocando una adición de información innecesaria y que perjudicaría el rendimiento de los algoritmos de aprendizajes supervisado y no supervisado.

\subsection{Ordinal encoder}

Ordinal encoder, es una simplicación de One Hot encoder, ya que, simplemente codifica las categorías con números en el conjunto $[0, m-1]$. Es decir, sea el vector $x$ de tamaño $n$ con $m$ categorías y sea $M$ el espacio de las posibles categorías con $M = [m_{1}, \cdots, m_{m}]$, y cuya codificación implica el vector $M' = [0, \cdots, m-1]$. $\forall $ elemento que $\in$ a $x$ se obtiene su codificación a partir del elemento $M'(M(m_{i}))$ que corresponde a la codificación de la categoría en el espacio $M$ \cite{pedregosa2011scikit}.

Es posible cuestionar esta metodología con respecto al orden en que trata las categorías, la representación de la información y el mantenimiento del significado de la data. Razón por la cual, se usa sólo en los casos en que la adición de múltiples descriptores empleando One Hot Encoder sea perjudicial a la hora de implementar modelos de clasificación o regresión, e inclusive, en la búsqueda de patrones.

\subsection{Frecuencias de residuos}

Una secuencia lineal de proteína, corresponde a un vector $v$ de tamaño $n$ donde cada elemento corresponde a un residuo que pertenece a la secuencia. El uso de esta información para alimentar modelos de clasificación o regresión conlleva la codificación de sus elementos. Sin embargo, a la hora de utilizar las codificaciones basadas en One Hot Encoder, el conjunto de datos no queda estándar en cuanto a sus dimensiones, ya que, el largo de las secuencias puede variar y a su vez, el número de columnas a agregar corresponde a $n \times 20$ dado a que son $n$ residuos y el espacio muestral $M$ es de tamaño 20 lo que genera un aumento considerable en la cantidad de dimensiones.

Con el fin de poder representar las secuencias lineales de proteínas, se idearon metodologías que consideran la frecuencia de aparición de los residuos en la secuencia, de tal manera, de poder codificarla en un vector de tamaño 20, donde cada elemento representa el número de incidencias del residuo dividido por el largo del vector. Así, cada elemento se encuentra en un rango $[0,1]$ donde 0 indica no incidencia del residuo y 1, incidencia total \cite{ozbudak2014protein}.

Expresado de forma matemática, sea $s$ una secuencia lineal de proteínas con $r$ residuos, su codificación se basa en la frecuencia de aparición del residuo en $s$, tal que, sea $R$ el espacio de los posibles residuos $r$ en $s$, se estima para cada $r_{j} \in\ R$ su frecuencia:

\begin{center}
	$frec(r_{j}) = \dfrac{cont(r_{j})\ if\  (r_{i}==r_{j})}{n}$
\end{center}

Finalmente, se tiene que cada residuo $r_{i}$ se representa en su valor de frecuencia $frec(r_{i})$, generando un set de datos de tamaños $s \times 20$ con $s$ secuencias representadas por un vector de tamaño 20.

Esta es una de las representaciones más utilizadas y más simples a la hora de codificar secuencias lineales de proteínas. Sin embargo, presenta diferentes problemas tales como:

\begin{itemize}
	
	\item Si los residuos se encuentran en proporciones similares, se generarán conjuntos de datos con atributos no informativos, ya que, disminuirá la varianza existente para dicho atributo, provocando una redundancia de datos.
	
	\item No considera información sobre los residuos asociados a propiedades fisicoquímicas, esto complica el hecho de representar un set de datos de secuencias o mutaciones, ya que no representa la realidad y sólo expone el comportamiento de las frecuencias de residuos, favoreciendo a aquellos con una mayor incidencia en sus elementos.
	
	\item La codificación por frecuencias es utilizada como un primer acercamiento a la representación del problema y principalmente perjudica a los modelos ya que puede generar atributos no informativos, como lo son los residuos sin incidencia, esto conlleva a modelos sobre ajustados y a creación de set de datos no informativos.
	
	\item No evalúa elementos relevantes a la caracterización de residuos claves, ambiente bajo el cual ocurren mutaciones o componentes adicionales que facilitarían una mayor comprensión del problema, ya que, sólo conocer las incidencias, proporciona un conocimiento sobre la moda y cuáles son los residuos más relevantes. No obstante, sólo permite inferir características, relacionadas a estos.
\end{itemize}

El uso de las frecuencias de residuos, es una de las primeras aproximaciones a la codificación de secuencias lineales de proteínas. No obstante, en todos los casos donde han sido utilizadas, se agrega información adicional, que permite comprender diferentes comportamientos y evalúa ciertas propiedades del entorno, razones por las cuales, se recomiendan utilizarlas en conjunto con otros descriptores. 

\subsection{Uso de propiedades fisicoquímicas}

El uso de propiedades fisicoquímicas para describir un residuo, es ampliamente empleado en la generación de descriptores para conjuntos de datos en ingeniería de proteínas \cite{capriotti2005mutant2, capriotti2008three}. Diversos enfoques y modelos han sido construidos o entrenados, contemplando información asociada a componentes termodinámicos del residuo, en particular, a la hora de describir residuos para evaluar cambios en la energía libre, relacionados a efectos en la estabilidad de una proteína \cite{ancien2018prediction, broom2017computational, 1gzp030}.

Se han reportado cerca de 570 propiedades fisicoquímicas que pueden ser utilizadas para describir un residuo en una secuencia lineal de proteínas, almanceadas en la base de datos AAIndex \cite{Kawashima2000}. A su vez, es posible caracterizar estos residuos empleando un conjunto de propiedades estructurales, termodinámicas e inclusive filogenéticas. Es decir, diferentes puntos de vista que permitan describir los residuos pertenecientes a una secuencia. Sin embargo, el hecho de seleccionar qué descriptores son relevantes y cuáles no, radica en un problema de evaluación de características, el cual es común, en el área de la minería de datos. 

Dado al gran conjunto de propiedades existentes y a la diversidad de descriptores que pueden ser utilizados para un conjunto de secuencias lineales de proteínas, es necesaria una selección correcta de las características, las cuales permitan formar set de datos informativos y con una correlación mínima entre sus elementos. 

Contemplando esta problemática, técnicas de reducción de dimensionalidad o análisis de características son las más utilizadas a la hora de seleccionar los descriptores más informativos para un conjunto de datos. No obstante, en ocasiones, el conocimiento sobre el problema es un factor relevante a considerar. 

Dada la relevancia de la selección de descriptores correctos para poder caracterizar y codificar secuencias lineales a partir de propiedades, se describen a continuación, algunas técnicas de reducción de dimensionalidad y análisis de características que pueden ser empleadas para dar solución a esta problemática.

\subsubsection{Técnicas de reducción de dimensionalidad}

Diferentes técnicas para el análisis de características y reducción de dimensionalidad han sido implementadas en el campo de minería de datos, con el fin de poder permitir la selección de descriptores informativos y sin contemplar conjuntos de datos altamente dimensionales. Dentro de las principales destacan: Análisis de correlación, mutual information, evaluaciones de características con respecto al entrenamiento de modelos empleando Random Forest, Análisis de componentes principales (PCA) y sus variantes como métodos lineales y reducción de dimensionalidad empleando métodos no lineales. 

\subsubsection{Análisis de correlación}

La correlación entre elementos, es posible definirla como una relación estadística entre dos variables aleatorias, asociado principalmente, a relacionales lineales entre ellas \cite{cohen2014applied}. Existen diferentes coeficientes de correlación, dentro de los cuales, el más conocido es Pearson. No obstante, se encuentran además: Spearman rank, kendall $\tau$. La formulación matemática de estos, fue expuesta en la sección \ref{desempeno}, en el apartado de análisis de desempeño de modelos de regresión.

De manera general, la correlación se relaciona con la covarianza entre las variables aleatorias \cite{cohen2014applied}, esto es: Sean $X$ e $Y$ variables aleatorias, el coeficiente de correlación $\rho$ entre ellas se define como:

\begin{center}
	$\rho X,Y = corr(X,Y) = \dfrac{cov(X,Y)}{\sigma_{Y}\sigma_{Y}} = \dfrac{E[(X - \mu_{X})(Y - \mu_{Y})]}{\sigma_{Y}\sigma_{Y}}$
\end{center}

$\rho \in [-1, 1]$ de tal forma que -1 indica una correlación inversa entre los elementos y 1 una relación positiva. En efectos prácticos, ambos valores denotan una dependencia entre los elementos, por lo que no son informativos y es posible eliminar uno de ellos, dada la redundancia de información. Valores cercanos a 0, indican que las variables no se encuentran correlacionadas.

Estos estudios se realizan para todas las variables en un conjunto de datos, expresándose en matrices de correlación, las cuales, pueden ser visualizadas mediante Heat Map en donde se expone la dependencia entre los atributos. Siendo una de las técnicas más utilizadas para la reducción de dimensionalidad, debido a su simpleza y a las ventajas que posee con respecto a la información que entrega sobre los atributos.

\subsubsection{Mutual information}

Es una medida de dependencia mutua entre dos variables aleatorias, permitiendo cuantificar la cantidad de información obtenida alrededor de una variable aleatoria vista desde otra. Representa un concepto más general que el análisis de correlación y se basa en la comparación de la similitud entre las distribuciones del conjunto de las variables independientes, con respecto al producto de éstas \cite{peng2005feature}.

Sean dos variables aleatorias $X$ e $Y$ con valores en el espacio $X \times Y$ y sea el conjunto $P_{(X,Y)}$ y las distribuciones marginales son $P_X$ y $P_Y$, respectivamente. El mutual information score para las variables $X$ e $Y$ es:

\begin{center}
	$I(X,Y) = D_{KL}(P_{(X,Y)} || P_X \otimes P_Y)$
\end{center}

Esto puede variar con respecto al tipo de variable a utilizar. En el caso de que pertenezcan a una distribución discreta el $I(X,Y)$ corresponde a:

\begin{center}
	$I(X,Y) = \sum_{y_{i} \in Y} \sum_{x \in X} P_{(x,y)} (x,y) \log (\dfrac{P_{(x,y)}(x,y)}{P_{x}(x)P_{y}(y)})$ 
\end{center}

Mientras que para el caso de variables con distribución continua se tiene:

\begin{center}
	$I(X,Y) = \int_{y} \int_{x} P_{(x,y)} (x,y) \log (\dfrac{P_{(x,y)}(x,y)}{P_{x}(x)P_{y}(y)})$
\end{center}

Como interpretación, se tiene que presenta un rango de valores no negativos, donde un valor 0 indica que no existe información mutua entre ambas características y a mayor valor más relación existe entre ambos elementos.

\subsubsection{Análisis espaciales de características}

En la sección \ref{rf}, se describió Random Forest como un algoritmo de aprendizaje supervisado, que permite entrenar modelos utilizando árboles de decisión para manipular las características, generando $n$ iteraciones, en donde se construyen $n$ árboles, con diferentes atributos y ejemplos. Esto, permite estabilizar las medidas de desempeño y generar modelos robustos y con probabilidades menores de sobreajuste.

Sin embargo, este método puede ser utilizado con el fin de evaluar las características, ya sea en torno a cuáles son las frecuencias con las que permite dividir de manera inicial el conjunto de datos o cuáles son las que permiten generar la clasificación o la asignación a un intervalo en términos de regresión \cite{saeys2008robust}. 

A partir de lo anterior, la profundidad de una característica utilizada como nodo de decisión en un árbol puede usarse para evaluar la importancia relativa de esa característica con respecto a la predictibilidad de la variable objetivo. Las características utilizadas en la parte superior del árbol contribuyen a la decisión de predicción final de una fracción mayor de las muestras de entrada. La fracción esperada de las muestras a las que contribuyen puede, por lo tanto, usarse como una estimación de la importancia relativa de las características \cite{granitto2006recursive}. 

La fracción de muestras a las que contribuye una característica se combina con la disminución de la impureza al dividirlas para crear una estimación normalizada del poder predictivo de esa característica \cite{saeys2008robust}.

\subsubsection{Métodos de reducción de dimensionalidad lineales}

Reducción de dimensionalidad, como su nombre lo sugiere, implica remover características o atributos del set de datos, con el objetivo de disminuir el número de dimensiones del conjunto de elementos. Esto permite descartar los atributos menos informativos o con menor relevancia, ya sea en términos de aporte a la varianza o relaciones con el resto de descriptores \cite{hinton2006reducing}.

Dentro de las principales técnicas de reducción lineales, se encuentran el Análisis de componentes principales (PCA) y sus variantes, tales como: Incremental PCA \cite{jolliffe2011principal}. 

Análisis de Componentes Principales (PCA, por sus iniciales en inglés), es una técnica estadística que permite la conversión de un conjunto de variables posiblemente correlacionadas a un conjunto de variables no correlacionadas linealmente. Estos elementos se denominan componentes principales. Su principal característica es que son ordenados de mayor a menor, según la varianza que entregan a los datos \cite{jolliffe2011principal}.

Intuitivamente, es posible pensar el PCA como un elipsoide n-dimensional de datos, donde cada eje del elipsoide representa un componente principal. Esto implica, que si algún eje del elipsoide es pequeño, la varianza correspondiente a lo largo de éste también lo es, por lo que omitir dicho eje no implica una pérdida importante de información, lo cual es denotado como la reducción de la dimensionalidad en base a los aportes a las varianzas que denotan cada componente \cite{wold1987principal}.

\paragraph{Definición\\\\}

Matemáticamente, es posible definir PCA como una transformación lineal ortogonal de los datos a un nuevo sistema de coordenadas, tal que, la mayor varianza por alguna proyección de los datos pasa a situarse en la primera coordenada (llamado el primer componente principal), la segunda mayor varianza en la segunda coordenada, y así sucesivamente.

Se considera un conjunto de datos, $X$, con una media empírica 0, donde cada una de las filas  representan ejemplos y las columnas características o atributos $p$.

La transformación está definida por un set de vectores de dimensión $p$ que poseen pesos denotados por $ w_{(k)} = (w_{1},...,w_{p})_{(k)}$, los cuales para cada vector $x_{i}$ en $X$ se operan para dar un vector con los componentes principales $ t_{(i)} = (t_{1},...,t_{k})_{(i)}$ el cual viene dado por $ t_{k(i)} = x_{i}  w_{k}$

De tal manera que las variables individuales de $t$ considerado sobre el conjunto de datos, sucesivamente heredan la varianza máxima posible de $x$, con cada carga del vector $w$.

El primer componente $ w_{{1}} $ debe satisfacer las siguientes características:

\begin{itemize}
	
	\item $w_{1} = \arg \max_{||w||=1} \{\sum_{i}(t_{(1)})^{2}_{i}\} = \arg \max_{||w||=1} \{\sum_{i}(x_{(i)}*w)^{2}\}$
	
	\item $w_{1} = \arg \max_{||w||=1} \{||Xw||^{2}\} = \arg \max_{||w||=1} \{w^{T}X^{T}Xw\}$
	
	\item $w_{1} = \arg \max \{\frac{w^{T}X^{T}Xw}{w^{T}w}\}$
	
\end{itemize}

Los \textit{k} restantes componentes son encontrados efectuando la extracción de los primeros \textit{k-1} componentes principales desde $x$:

\begin{center}
	$\hat{x}_{k} = x - \sum_{\delta=1}^{k-1}Xw_{(\delta)}w^{T}_{(\delta)}$
\end{center}

A su vez, para encontrar el vector de carga, es necesario extraer la varianza máxima del nuevo set de datos, tal que:

\begin{center}
	$w_{(k)} = \arg \max_{||w||=1} \{||\hat{x}_{k}w||^{2}\} = \arg \max \{\frac{w^{T}X^{T}\hat{X}^{T}_{k}\hat{x}_{k}w}{w^{T}w}\}$
\end{center}

La matriz de covarianza juega un rol fundamental en este análisis, cuyo valor entre dos componentes principales viene dado por:
\begin{center}
	$Q(PC_{j}, PC_{k}) \infty (Xw_{(j)})^{T} * (Xw_{(k)})$\\
	$Q(PC_{j}, PC_{k}) = w_{(j)}^{T}X^{T}Xw_{(k)}$\\
	$Q(PC_{j}, PC_{k}) = w_{(j)}^{T}\lambda_{(k)}w_{(k)}$\\
	$Q(PC_{j}, PC_{k}) = \lambda_{(k)}w_{(j)}^{T}w_{(k)}$	
\end{center}


La principal característica que define al PCA es que es una técnica comúnmente utilizada para la reducción de la dimensionalidad, esto viene dado por la transformación que se genera, $ T = xw $ donde cada vector $ x_{(i)} $ existente en un espacio de coordenadas de variables $p$, es representado por un nuevo espacio, en el cual las variables no se encuentran correlacionadas. Sin embargo, si se utilizan $ L $ componentes principales para así utilizar los primeros $ L $ vectores de carga se obtiene una transformación truncada $ T_{L} = XW_{L} $, de tal manera que la matriz $ T_{L} $ posee los $n$ ejemplos originales. 

No obstante, sólo posee $ L $ características que definen el set de datos, de tal manera que dicha transformación es posible expresarla como:

$ t = W^{T}x$, donde $ x \in R^{p}, t \in R^{L}$, para las cuales las columnas $ p \times L $ de la matriz $ W $ forman una base ortogonal de las $ L $ características, de esta manera, al basarse en la construcción con sólo $ L $ columnas se maximiza la varianza original de los datos y se minimiza el error cuadrático tal que: 

\begin{center}
	$|| TW^{T} - T_{L}W_{L}^{T}||^{2}_{2} = ||X - X_{L}||_{2}^{2}$
\end{center}

Normalmente esta reducción es usada para el manejo de set de datos de alta dimensionalidad.

PCA presenta algunas variaciones en sus definiciones o en el uso de los datos, tales como: Incremental PCA. Este método, representa una ventaja computacional en cuanto al coste de memoria y a la manipulación de set de datos altamente dimensionales \cite{artac2002incremental}.


\subsubsection{Métodos de reducción de dimensionalidad no lineales}


\subsection{Codificación de residuos con adición de información de su entorno}

Adicional a las técnicas explicadas previamente con respecto a las codificaciones existentes, en algunos casos, no sólo basta con una única codificación del residuo, si no, que es relevante adicionar información que puede ser importante para describir los residuos. Normalmente, junto con las codificaciones basadas en propiedades fisicoquímicas, se emplean técnicas que permitan describir el ambiente bajo el cual se encuentre el residuo \cite{masso2008accurate}.

En la gran mayoría de los casos, se adiciona información de los residuos cercanos al residuo de interés, esto depende del tipo de datos bajo el cual se esté trabajando, es decir, si son secuencias lineales o son estructuras de proteínas en formato PDB \cite{capriotti2008three, capriotti2005mutant2}. 

Para el caso de que sean secuencias lineales, sea $s$ secuencia de residuos de tamaño $n$ y sea $r_{i}$ el residuo de interés a evaluar su ambiente. Se crea una ventana de tamaño $n'$ que contempla la cantidad de residuos $r_j$ cercanos al residuo $r_{i}$, de tal manera que se crea un nuevo sub conjunto $s'$ de datos de tamaño $2n'$ con $n'$ residuos a la izquierda y $n'$ a la derecha. El cual normalmente es codificado empleando binarización de elementos, así, en algunas ocasiones, a cada residuo, se le adicionan 20 descriptores que permiten indicar la ausencia o presencia de residuos cercanos a su entorno y el cual se completa con el conjunto de residuos $s'$ \cite{capriotti2008three}.

Cuando se manejan estructuras de proteínas en formato PDB, la codificación y la evaluación del ambiente es similar. Sin embargo, en vez de utilizar una ventana de tamaño $n'$ se utiliza un radio espacial de valor $x$ para el cual, se toma el residuo y se estiman las distancias de los elementos cercanos, ya sea entorno a los carbonos $\alpha$ o a otros elementos. Esto, a diferencia de las secuencias lineales, permite adicionar información sobre las propiedades de distancia, ángulos y conformación de estabilidad por interacciones electrostáticas débiles que pueden generarse a partir de la proximidad de los elementos. No obstante, es una inferencia de su uso y se requieren de diferentes tipos de elementos que permitan caracterizar los eventos asociados al ambiente estructural asociado al residuo \cite{capriotti2008three}.

Actualmente, el uso de codificaciones mediante propiedades fisicoquímicas y el empleo de información adicional basada en descriptores de ambientes, es una de las metodologías más utilizadas a la hora de generar set de datos relacionados a mutaciones. Sin embargo, debido a que sólo se considera distancia, la binarización de los elementos no se ve afectada por sustituciones en residuos lejanos al lugar de ocurrencia, lo que denota la necesidad de idear metodologías que permitan contemplar el aporte completo de residuos a la caracterización de propiedades y cómo sustituciones puntuales afectan enormemente a residuos de interés. Una de las formas en las que se ha intentado dar solución a esta problemática, es modelar las propiedades fisicoquímicas de los residuos de las secuencias, a partir del uso de transformaciones de Fourier y en particular, empleando algoritmos relacionados a dichos conceptos, que aprovechen las ventajas referidas a la manipulación de espacios de frecuencias por sobre elementos temporales.

\section{Transformaciones de Fourier}

\subsection{Transformada rápida de Fourier (FFT)}

\subsection{Uso de Transformadas de Fourier en digitalización de propiedades fisicoquímicas}

\section{Hipótesis}

Dada la problemática existente sobre cómo representar conjuntos de secuencias lineales con el fin de poder desarrollar modelos de clasificación/regresión o identificación de patrones asociados a residuos claves que brindan la propiedad fisicoquímica. Y, en consideración de los diferentes usos que entrega las transformadas de Fourier, se plantea la hipótesis de este capítulo.

La codificación de secuencias lineales empleando espectros de frecuencia, basados en las propiedades fisicoquímicas de los residuos pertenecientes, permite generar descriptores que faciliten el aprendizaje de predictores de variantes enfocados a diferentes respuestas de interés? A su vez, los patrones de frecuencia, pueden ser asociados a residuos claves en familias de variantes?

Si bien, en el planteamiento de la hipótesis se exponen dos interrogantes, la interrogante en sí, se centra a los posibles usos que pueda tener el uso de los espectros de frecuencia en el estudio de variantes, identificación de patrones, residuos relevantes, etc.

\section{Objetivos}

En base a la hipótesis planteada y con el fin de responder a los planteamientos e interrogantes expuestas. Se detallan el objetivo general y los objetivos específicos.

\subsection{Objetivo general}

Diseñar e implementar metodología de codificación y digitalización de propiedades fisicoquímicas en secuencias lineales de proteínas, con el fin de poder ser utilizadas en identificación de patrones por medio de técnicas de clustering o desarrollo de predictores basados en algoritmos de aprendizaje supervisado.

\subsection{Objetivos específicos}

A partir del objetivo general, nacen los siguientes objetivos específicos.

\subsection{Preparar, manipu}
\section{Metodología}


 