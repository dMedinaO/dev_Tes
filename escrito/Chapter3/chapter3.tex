%!TEX root = ../thesis.tex
%*******************************************************************************
%****************************** Third Chapter **********************************
%*******************************************************************************
\chapter{Digitalizando propiedades fisicoquímicas de proteínas a partir de su secuencia lineal}

% **************************** Define Graphics Path **************************
\ifpdf
    \graphicspath{{Chapter3/Figs/Raster/}{Chapter3/Figs/PDF/}{Chapter3/Figs/}}
\else
    \graphicspath{{Chapter3/Figs/Vector/}{Chapter3/Figs/}}
\fi

Existen cerca de X proteínas reportadas en las bases de datos. Sin embargo, sólo un número limitado de ellas presentan cristal o estructura tridimensional reportada, lo cual dificulta diferentes estudios posibles a la hora de analizar mutaciones y cambios conformacionales que conlleva dicho cambio por medio de técnicas bioinformáticas que requieren la estructura de la proteína.
 